\documentclass[openany,italian]{book}
\usepackage[letterpaper,top=2cm,bottom=2cm,left=3cm,right=3cm,marginparwidth=1.75cm]{geometry}
\usepackage[utf8]{inputenc}
\usepackage[italian]{babel}
\usepackage[Lenny]{fncychap}
\usepackage{url}
\usepackage{lipsum}
\usepackage{graphicx}
\usepackage[version=4]{mhchem}
\usepackage{amsmath}
\usepackage{mathtools,amsfonts,amssymb}
\usepackage{tikz}
\usetikzlibrary{decorations.markings}
\usetikzlibrary{arrows.meta}
\usepackage{rotating}
\usetikzlibrary{calc}

\newcommand{\tikzmark}[1]{\tikz[overlay,remember picture] \node (#1) {};}

\usepackage{bbold}
\usepackage{booktabs}
\usepackage[arrowdel]{physics}
\usepackage{cancel}
\usepackage{float}
\usepackage{enumitem}
\usepackage{hyperref}
\hypersetup{
    colorlinks,
    citecolor=black,
    filecolor=black,
    linkcolor=black,
    urlcolor=black
}

\makeatletter
\newcommand{\lambdabar}{{\mathchoice
  {\smash@bar\textfont\displaystyle{0.25}{1.2}\lambda}
  {\smash@bar\textfont\textstyle{0.25}{1.2}\lambda}
  {\smash@bar\scriptfont\scriptstyle{0.25}{1.2}\lambda}
  {\smash@bar\scriptscriptfont\scriptscriptstyle{0.25}{1.2}\lambda}
}}
\newcommand{\smash@bar}[4]{%
  \smash{\rlap{\raisebox{-#3\fontdimen5#10}{$\m@th#2\mkern#4mu\mathchar'26$}}}%
}
\makeatother

\newcommand{\A}{\text{Å} \hspace{0.1mm}}

\newcommand{\E}{È \hspace{0.1mm}}

\newcommand\parallelo{/\!/}

\setlength\parindent{0pt}

\DeclareUnicodeCharacter{2212}{-}

\newlist{Properties}{enumerate}{2}
\setlist[Properties]{label=(p.\arabic*),itemindent=*}

\begin{document}

\title{Appunti di Meccanica Quantistica Avanzata}
\author{Marco Gorgone}

\thispagestyle{empty}
\begin{center}

\begin{minipage}[c]{0.45\textwidth}
\begin{flushleft}
\includegraphics[width=0.8\textwidth]{build/logo_unict_orizzontale.png}
\end{flushleft}
\end{minipage}
\hfill
\begin{minipage}[c]{0.45\textwidth}
\begin{flushright}
\includegraphics[width=\textwidth]{build/logo_dfa_orizzontale}
\end{flushright}
\end{minipage}\\
\medskip
\hbox to \textwidth{\hrulefill}

\vfill
\vfill


\uppercase{\sc{ \Large{\textbf{Use of quantum algorithm to solve real gas}}}}\\

\vfill
\begin{center}
    \large{Use of quantum algorithm to solve real gas}
    
    \vspace{0.5cm}
    
    \large{Marco Gorgone}    
\end{center}

\vfill
\vfill
\hbox to \textwidth{\hrulefill}
{\sc anno 2023}
\end{center}

\tableofcontents

%\include{introduzione/intro}

\end{document}